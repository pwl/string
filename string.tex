\documentclass[
a4paper,% paper dimensions
10pt,% font size
titlepage,% print title on a single page
%twocolumn,% text in two columns on a single page
twoside% two sided printing (page numbers on outer edge)
]{article}


% \pagestyle{myheadings}

\usepackage{amsmath}
\usepackage{amssymb}
\usepackage{amscd}
% \usepackage[polish]{babel}
% \usepackage{polski}
\usepackage[utf8]{inputenc}
\usepackage[pdftex]{hyperref}
% \usepackage{graphicx}
% \usepackage{titlesec}
% \usepackage{afterpage}
% \usepackage[section]{placeins}


\begin{document}

\subsection{Recursive formulation}
We start with the equation
\begin{gather}
  \label{eq:wave0}
  (\rho+m\delta(x-L))\partial_{tt}\psi=\partial_{xx}\psi,
\end{gather}
which after the substitutions
\begin{gather}
  \label{eq:subs0}
  x\rightarrow x'=\frac{m}{\rho} x         \\
  t\rightarrow t'=\frac{m}{\sqrt{2\rho}} t \\
  L'=\frac{\rho L}{m}
\end{gather}
becomes
\begin{gather}
  \label{eq:wave0}
  (1+\delta(x-L))\partial_{tt}\psi=\frac{1}{2}\partial_{xx}\psi,
\end{gather}
with primes omitted, so the problem has one parameter (which can be
also shown using dimensional analysis). Futher, the most general
ansatz for the escaping wave to the right of the point mass is
\begin{gather}
  \label{eq:ansatz0}
  \psi(x,t)=
  \begin{cases}
    f(x-t)+g(x+t) & x\in[0,L] \\
    h(x-t)        & x\ge L,
  \end{cases}
\end{gather}
with conditions
\begin{gather}
  \label{eq:boundary}
  \psi(0,t) = 0\quad\text{zero at the boundary}\\
  \label{eq:continuity}
  \psi(L-,t)=\psi(L+,t)\quad\text{continuity}.
\end{gather}
which become
\begin{gather}
  \label{eq:ansatz0}
  \psi(x,t)=
  \begin{cases}
    -g(t-x)+g(x+t) & x\in[0,L] \\
    h(x-t)         & x\ge L.
  \end{cases}\\
  \label{eq:cont0}
  -g(t-L)+g(L+t)=h(L-t)\quad\text{continuity}
\end{gather}

Integrating \eqref{eq:wave0} we obtain
\begin{gather}
  \partial_{tt}\psi(L,t)=\frac{1}{2}(\partial_x\psi(L+,t)-\partial_x\psi(L-,t)).
\end{gather}
which glues together the solutions from different intervals along with
the continuity condition \eqref{eq:continuity}.

The preceding statements give rise to the following equation for $g$
\begin{equation}
  \begin{split}
    -g''(t-L)+g''(t+L)&=\frac{1}{2}(h'(L-t)-g'(t-L)-g'(t+L))\\
    &=\frac{1}{2}(g'(t-L)-g'(t+L)-g'(t-L)-g'(t+L))\\
    &=-g'(t+L),
  \end{split}
\end{equation}
and after the shift in $t$ and with $T:=2L$ we get
\begin{gather}
  g''(t+T)+g'(t+T)=g''(t)
\end{gather}
integration gives us
\begin{gather}
  g'(t+T)+g(t+T)=g'(t)+(g'(T)-g'(0)+g(T)).
\end{gather}
Adding a constant to $g$ does not change the equations, so we can
add such constant in order to cancel the $(g'(T)-g'(0)+g(T))$ term,
and we are then given the final form of equation for $g$
\begin{gather}
  \label{eq:dde}
  g'(t+T)+g(t+T)=g'(t),
\end{gather}
which is a kind of delayed differential equation, and for which one
can iteratively find a solution for arbitrary large times given the
function values in an interval $[a,a+T]$ where $a$ is an arbitrary
constant for convinience we shall fix $a=0$. Now we shall introduce a
family of functions $g_n\in\mathcal{C}([0,T])$ defined as
\begin{equation}
    g(t+nT)=g_n(t)\quad t\in[0,T]
\end{equation}
so the equation \eqref{eq:rec0} becomes
\begin{gather}
  \label{eq:rec0}
  \begin{split}
    &g'_{n+1}(t)+g_{n+1}(t)=g'_n(t),\\
    &g_{n+1}(0)=g_n(T)\quad\text{continuity condition}.\\
    &\text{with}\quad g_0(t)\quad\text{initial conditions}
  \end{split}
\end{gather}
The above equation can be solved by iteration
\begin{gather}\label{eq:defL}
  g_{n+1}(t)=(Lg_n)(t)=g_n(t)-e^{-t}\int_0^t e^s g_n(s)ds+e^{-t}(g_n(T)-g_n(0)).
\end{gather}
With $L:\mathcal{C}([0,T])\rightarrow \mathcal{C}([0,T])$. The
eigenvectors of $L$ are the soultions of the equation \eqref{eq:rec0}
with $g_{n+1}=\lambda g_n=\lambda g_\lambda$
\begin{gather}
  \lambda(g'_\lambda(t)+g_\lambda(t))=g'_\lambda(t),\\
  \lambda g_\lambda(0)=g_\lambda(T)\quad\text{quantization}.
\end{gather}
from which we obtain
\begin{gather}
  g_\lambda(t)=e^{\frac{\lambda}{\lambda+1}t},\\
  \lambda=e^{\frac{\lambda}{\lambda+1}T}\quad\text{quantization}.
\end{gather}
The last term in \eqref{eq:defL} may look suspicious due to its form
(explicit function times constant) but for eigenvectors it is simply
zeros altogether with lower boundary of integration according to the
quantization condition. The quantization for $\lambda$ gives rise to
the quasi-normal modes of equation \eqref{eq:wave0}.

\subsection{Quasinormal Modes - asymptotic expansion}
The transcendental equation for $\lambda$ can be written as
\begin{gather}\label{eq:qm0}
  \frac{\beta}{\beta-i}=e^{i\beta T}
\end{gather}
where $\lambda=\frac{\beta}{\beta-i}$. We shall split $\beta$ into its
real and imaginary part: $\beta=\Omega+i\Gamma$.  Taking the absolute
value of \eqref{eq:qm0} we obtain the following equation
\begin{gather}\label{eq:qm1}
  \frac{\Omega^2+(\Gamma-1)^2}{\Omega^2+\Gamma^2}=e^{2\Gamma T}
\end{gather}
Let's assume $\Gamma<0$, then from the equality
\begin{gather}
  e^{2\Gamma T}-1=\frac{1-2\Gamma}{\Omega^2+\Gamma^2}
\end{gather}
contradiction appears and so we deduce $\Gamma>0$, which in turn leads
to the fact that $\Gamma<\frac{1}{2}$. From the fact $\Gamma>0$ one
can conclude that $L$ is bounded, and thus continous on the space spanned
by its eigenvectors because
\begin{gather}
|\lambda|^2=e^{-2\Gamma T}<1.
\end{gather}
Equation
\eqref{eq:qm1} can be solved for $\Omega^2$
\begin{gather}
  \begin{split}
    \Omega^2&=\frac{1-2\Gamma-\Gamma^2(e^{2\Gamma T}-1)}{e^{2\Gamma T}-1}\\
    &=\frac{1}{2\Gamma T}-\frac{2+T}{2T}+\Gamma\frac{6+T}{6}+\mathcal{O}(\Gamma^2).
  \end{split}
\end{gather}
We shall introduce a new parameter $\eta:=\frac{1}{\sqrt{\Gamma}}\in[\sqrt{2},\infty]$ obtaining
the parametrization $\beta(\eta)$
\begin{gather}\label{eq:qmparam}
  \Gamma=\frac{1}{\eta^2},\\
  \Omega=\\
  \pm\sqrt{\frac{\eta^4-2\eta^2-(e^{2T/\eta^2}-1)}{\eta^4(e^{2T/\eta^2}-1)}}=\\
  \pm(\frac{1}{\sqrt{2 T}}\eta-\frac{\sqrt{2T}(2+T)}{4T}\frac{1}{\eta}+\mathcal{O}(\eta^{-3})),\\
  i\beta=\pm(\frac{i}{\sqrt{2 T}}\eta-\frac{i\sqrt{2T}(2+T)}{4T}\frac{1}{\eta})-\frac{1}{\eta^2}+\mathcal{O}(\eta^{-3})
\end{gather}
We should now calculate $g(t)=(L^n h_0)(t-nT)$ for
\begin{gather}
  h_0(t)=\sum_\lambda a_\lambda g_\lambda(t).
\end{gather}
We have
\begin{gather}
  (L^n h_0)(t)=\sum_\lambda \lambda^n a_\lambda g_\lambda(t).
\end{gather}
Utilizing quantization condition for $\lambda$ we obtain
\begin{gather}\label{eq:series}
  (L^n h_0)(t) =\sum_\lambda a_\lambda e^{\frac{\lambda}{\lambda+1}(t+nT)}
\end{gather}
Inserting the \eqref{eq:qmparam} to \eqref{eq:series} yields
\begin{gather}
  \begin{split}
    (L^n h_0)(t-nT) &=\sum_\eta a_{\lambda(\eta)} e^{i\beta(\eta)t}\\
    &\approx \int_{\eta_0}^\infty a_{\lambda(\eta)}\exp((ib\eta-\frac{ic}{\eta}-\frac{1}{\eta^2})t)d\eta
  \end{split}
\end{gather}
The remaining problem is the asymptotic form of $a_{\lambda}$ for
reasonable initial data (which cannot be simply computed using scalar
product, because eigenvectors are not orthogonal in their current
form) and integration of the above integral to get energy leak in time
(see \eqref{eq:enleak}). The integration should be doable using the
method of steepest descent or some other method for highly oscillatory
integrands assuming $t\gg 1$. After this the asymptotic expansion in
time should be easily obtained, and so the energy leak.

% The integration can be done using the method
% of steepest descent after expanding functions under the integrand as
% asymptotic series for $t$. Namely introducing the substitutions
% $f(\eta)=f(\eta_0)-w^2$, where $\eta_0$ is an appropriately chosen
% saddle point, calculating $\eta(w)$ and $d\eta=\frac{d\eta(w)}{dw}dw$
% as well as $a_{\lambda(\eta)}$, expanding
% $a_{\lambda(\eta)}\frac{d\eta}{dw}$ as a series in time obtaining the
% series of integrals which one shall then calculate term by term. The
% cructial point is that calculation of integrals is not necessary to
% obtain the leading term of time dependancy.


% We are looking for $(L^ng_0)$ for $n\rightarrow\infty$.

\subsection{Eigenvalues of $L$}

Equation
\begin{gather}\label{eq:beta}
  \frac{\beta}{\beta-i}=e^{i\beta T}
\end{gather}
defines the discrete set ${\beta_i}$ with elements $\beta_i$

\subsection{Energy}
The energy of the whole string is given by the functional
\begin{gather}
  \begin{split}
    E(\psi)&=\frac{1}{2}\int_0^\infty dx ((1+\delta(x-L))\psi_t^2+\psi_x^2)\\
    &=\frac{1}{2}\int_0^L dx(\psi_t^2+\psi_x^2)+\frac{1}{2}\int_L^\infty dx(\psi_t^2+\psi_x^2)+\frac{1}{2}\psi_t^2\bigg|_{x=L}\\
    &=E_{in}(\psi)+E_{esc}(\psi)
  \end{split}\\
  E_{in}(\psi)=\frac{1}{2}\int_0^L dx(\psi_t^2+\psi_x^2)+\frac{1}{2}\psi_t^2\bigg|_{x=L}\\
  E_{esc}(\psi)=\frac{1}{2}\int_L^\infty dx(\psi_t^2+\psi_x^2)
\end{gather}
% The whole energy of the string is conserved so it is enough to find
% $E_{esc}(\psi)$. Using the relations \eqref{eq:ansatz0} and
% \eqref{eq:cont0} and \eqref{eq:dde} we obtain
% \begin{gather}
%   \begin{split}
%     E_{esc}(\psi)&=\int_L^\infty dx((g'(t-x+T)-g'(t-x))^2=\int_L^\infty dx(g(t-x+T))^2=\\
%     &=\sum_{n=0}^\infty\int_{L+nT}^{L+nT+T}dx(g(t-x+T))^2=\\
%     &=\sum_{n=0}^\infty
%   \end{split}
% \end{gather}
\begin{gather}
  \begin{split}\label{eq:enleak}
    \partial_t E_{in}(\psi)&=\int_0^L(\partial_x(\psi_t\psi_x))+\psi_{tt}\psi_t\bigg|_{x=L}=\\
    &=(\psi_x+\psi_{tt})\psi_t\bigg|_{x=L}=\\
    &=(g'(t-L)+g'(t+L)-g''(t-L)+g''(t+L))(g'(t+L)-g'(t-L))\\
    &=(g'(t-L)+g'(t+L)-g'(t+L))(g'(t+L)-g'(t-L))\\
    &=-g'(t-L)g(t+L)\\
    &=-(g'(t+L)+g(t+L))g(t+L)\\
    &=-\frac{1}{2}\partial_tg(t+L)^2-g(t+L)^2
  \end{split}
\end{gather}
So the highest order dependency on $t$ will be the $g(t+L)^2$ term (assuming $g\sim t^{-\alpha}$), so
we can write that for large times we have
\begin{gather}
  \partial_t E_{in}(\psi)\sim~-g^2(t+L)<0
\end{gather}

% Substituting
% \begin{gather}
%   g_n(t)=\frac{1}{\sqrt{2\pi}}\int_{\mathbb{R}}\hat{g_n}(\omega)e^{i\omega t}d\omega
% \end{gather}
% one gets
% \begin{gather}
%   g_{n+1}(t)=\frac{1}{\sqrt{2\pi}}\int_{\mathbb{R}}\hat{g_n}(\omega)\bigg[
%   \frac{i\omega}{1+i\omega}e^{i\omega t}+
%   e^{i\omega T-t}-
%   \frac{i\omega}{1+i\omega}e^{-(i\omega+1)t}
%   \bigg]d\omega.
% \end{gather}
% by multiplying by $e^{-i\omega't}/\sqrt{2\pi}$ and integrating
% $\int_0^{T}dt$ (this assumes that $g_n$ is zero everywhere except the
% interval $[a,a+T]$ where we have chosen $a$ to be $0$) one obtains
% \begin{gather}
%   \hat{g_{n+1}}(\omega')=\hat{g_n}(\omega')\frac{i\omega'}{1+i\omega'}+
%   \frac{1}{2\pi}\int_{\mathbb{R}}\hat{g_n}(\omega)\frac{e^{i\omega T}-e^{-T-iT(\omega'-\omega)}}{1+i\omega'}
% \end{gather}

\end{document}
